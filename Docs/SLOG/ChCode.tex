\newpage

\section{Implementacja}
\label{Sec_Implementacja}

\subsection{Zgłaszanie logów w kodzie}
Wszystkie logi w programie muszą być zapisywane poprzez makra \texttt{LOG\_ERROR}, \texttt{LOG\_WARNING}, \texttt{LOG\_INFO}, \texttt{LOG\_DEBUG} (patrz \texttt{slog.h}).

Na początku każdego pliku w którym będą zgłaszane logi należy dodać definicję:
\begin{lstlisting}[language=c++]
#define SLOGNAME FILENAME
\end{lstlisting}
, gdzie \texttt{FILENAME} jest nazwą pliku.

Poziom zapisywanych logów można ustawić definiując odpowiednią wartość w \texttt{LOG\_LEVEL} z pliku \texttt{slog.h}.

\subsection{Generowanie mapy i definicji}
\TODO