\newpage

\section{Dane binarne}
\label{Sec_DaneBinarne}

Wpisy logów przechowywane są w buforze jeden za drugim, każdy wpis składa się z od dwóch do pięciu słów \texttt{uint32\_t}:

\begin{longtable}{|p{\BWIDTH}|p{\OWIDTH}|}
	\hline
	\textbf{Bajt} & \textbf{Opis}\\ 
	\hline
	\hline
	\texttt{0 -- 3} & \textbf{log\_id}: identyfikator zdarzenia zapisany jako \texttt{uint32\_t}. Wartości \texttt{0x00000000} i \texttt{0xFFFFFFFF} oznaczają wpis pusty, nieposiadający innych pól niż \textbf{log\_id}, należy go zignorować i przejść jedno słowo dalej w pamięci.\\
	& \begin{tabular}{|p{\IBWIDTH}|p{\IOWIDTH}|}
		\hline
		\textbf{Bit} & \textbf{Opis}\\ 
		\hline
		\hline
		0 -- 1 & \textbf{Args}: liczba argumentów (\texttt{uint2}).\\
		\hline
		2 -- 3 & \textbf{Level}: poziom zdarzenia (\texttt{uint2}). 0 -- error, 1 -- warning, 2 -- info, 3 -- debug.\\
		\hline
		4 -- 28 & \textbf{Id}: id zdarzenia (\texttt{uint25}). Wartości puste i pełne (same zera lub same jedynki) nie mogą być wykorzystywane jako \textbf{Id}\\
		\hline
		29 -- 31 & \textbf{CSum}: suma kontrolna (\texttt{uint3}), liczba niezerowych bitów w \textbf{log\_id} (z wyłączeniem \textbf{CSum}).\\
		\hline
	\end{tabular}\\
	\hline
	\texttt{4 -- 7} & \textbf{Timestamp}: znacznik czasu zdarzenia [$\mu s$] (\texttt{uint32\_t}) \\
	\hline
	\texttt{8 ...} & \textbf{Arg[x]}: kolejno zapisane argumenty zdarzenia (\texttt{uint32\_t}), łącznie \textbf{Args} argumentów.
	\\
	\hline
\end{longtable}

W miarę ubywania wolnej przestrzeni w buforach logów najstarsze wpisy mogą być wymazywane bez żadnego powiadomienia / oznaczenia. Warunkiem jest zagwarantowanie, że wpisy będą wyrównane do początku bufora (inaczej dane byłyby nie do odczytania).