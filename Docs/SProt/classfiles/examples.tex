\section{Lateksowe przykłady}

\TODO
\CONF

\definecolor{orange}{RGB}{0,140,140}\pagecolor{orange}

\vspace{1cm}
\pabox{wyświetla tekst arg1 w ramce}
dupa\\
\papbox{Grafika/test}{ wyświetla tekst arg2 w ramce z obrazem arg1}
dupa\\
\paabox{wyświetla tekst arg1 w ramce z obrazem UWAGA!}
dupa\\
\begin{cminipage}
	\texttt{[0x00]\\}
	\texttt{[0xFF]\\}
	\texttt{[bajt komendy]\\}
	\texttt{[rozmiar ładunku w bajtach]\\}
	\texttt{....ładunek\\}
	\texttt{[zanegowany bajt komendy]\\}
\end{cminipage}
dupa\\

\subsection{Grafika}

\begin{figure}[h]
	\centering
	\includegraphics[width=0.6\linewidth]{Grafika/test}
	\caption{Układ elektroniki.}
	\label{fig:uklad}
\end{figure}

\subsection{Tabele}

\begin{table}[!h]
	\centering
	\begin{tabular}{|l|l|l|l|}
		\hline
		\boldmath$t\ [^oC]$ & \boldmath$V_{br}\ [V]$ & \boldmath$hPE\ [mV]$ & \boldmath$fPE\ [kHz]$\\ 
		\hline
		\hline
		$-5$ & $50,9$ & $18$ & $5,3$\\
		\hline
		$28$ & $52,7$ & $17$ & $98$ \\
		\hline
		$50$ & $53,7$ & $15$ & $450$ \\
		\hline
	\end{tabular}
	\caption{Pomiar parametrów MPPC w funkcji temperatury.}
	\label{tab:temp}
\end{table}

\begin{table}[!h]
	\centering
	\begin{tabular}{|l|l|l|l|l|l|l|l|l|}
		\hline
		\textbf{HV} & \textbf{f$_0$[Hz]} & \textbf{Tło[Hz]} & \multicolumn{2}{c|}{\textbf{Am-241 (moneta)}} & \multicolumn{2}{c|}{\textbf{Am-241 (granat)}} & \multicolumn{2}{c|}{\textbf{Cs-137}}\\ 
		\hline
		[V] & [Hz] & [Hz] & [Hz] & [\% tła] & [Hz] & [\% tła] & [Hz] & [\% tła]\\
		\hline
		\hline
		58 & $10$ & $9,8$ & $9,8$ & $100$ & $42$ & $428$ & $699$ & $7132$ \\
		\hline
		58 & $50$ & $58$ & $70,5$ & $122$ & $163$ & $281$ & $907$ & $1563$ \\
		\hline
		58 & $300$ & $342$ & $357$ & $104$ & $537$ & $157$ & $1342$ & $392$ \\
		\hline
	\end{tabular}
	\caption{{Wyniki badań plastiku.}}
	\label{tab:lab2w}
\end{table}

\begin{table}[!h]
	\centering
	\begin{tabular}{|l||l|l|l|}
		\hline
		\backslashbox{\boldmath$V_{in}$}{\boldmath$f$} & \boldmath$1\ Hz$ & \boldmath$10\ Hz$ & \boldmath$100\ Hz$\\ 
		\hline
		\hline
		\boldmath$58\ V$ & $2,67\ mV$ & $2,27\ mV$ & $1,87\ mV$\\
		\hline
		\boldmath$64\ V$ & $78\ mV$ & $59,5\ mV$ & $40,5\ mV$ \\
		\hline
	\end{tabular}
	\caption{Zbiór poziomów wyzwalania dla których osiąga się wymaganą częstotliwość zliczeń, $f$ -- docelowa częstotliwość zliczeń, $V_{in}$ -- napięcie zasilania MPPC.}
	\label{tab:labdd}
\end{table}

\begin{table}[H]
	\centering
	\begin{tabularx}{\linewidth}{|>{\setlength\hsize{1.05\hsize}}X|%
			>{\setlength\hsize{0.3\hsize}}X|>{\setlength\hsize{1.65\hsize}}X|}
		\hline
		\textbf{Nazwa} & \textbf{Wartość} & \textbf{Opis} \\
		\hline
		\hline
		\texttt{evtE\_NOACKNOWLEDGE\_STAGE\_1} & \texttt{0x84} & Wysłane, gdy zaadresowane w transakcji \ urządzenie nie potwierdziło zaadresowania w pierwszej części transakcji. \textbf{Informacja:} Zazwyczaj jeżeli urządzenie jest zajęte zamiast nie potwierdzać zaadresowania zatrzyma ono linię \texttt{SCL} w stanie niskim co ewentualnie może doprowadzić do \texttt{evtE\_TIMEOUT}.  \\ 
		\hline
		\texttt{evtE\_NOACKNOWLEDGE\_STAGE\_2} & \texttt{0x85} & Jak wyżej, ale dotyczy drugiej części transakcji.  \\ 
		\hline
	\end{tabularx}
	\caption[as]{Kody błędów protokołu wymiany danych.}
	\label{fig:Bledy}
\end{table}

\begin{longtable}{|p{0.1\textwidth}|p{0.8\textwidth}|}
	\hline
	\textbf{Bajt} & \textbf{Opis}\\ 
	\hline
	\hline
	\texttt{0} & \textbf{Start}: stała wartość rozpoczynająca pakiet, zawsze równa \texttt{0x5A}\\
	\hline
	\texttt{1} & \\
	& \begin{tabular}{|p{0.07\textwidth}|p{0.63\textwidth}|}
		\hline
		\textbf{Bit} & \textbf{Opis}\\ 
		\hline
		\hline
		0 -- 6 & \textbf{Bytes}: liczba bajtów ładunku, od 0 do 64.\\
		\hline
		7 & \textbf{CmdH}: najstarszy bit komendy.\\
		\hline
	\end{tabular}\\
	\hline
	\texttt{2} & \textbf{CmdL}: młodszy bajt komendy.\\
	& \textbf{Cmd = CmdL | CmdH<<8}: Komanda pakietu.
	\\
	\hline
	\texttt{3} ... & \textbf{Ładunek}: ładunek komendy, liczba bajtów w ładunku zapisana w polu \textbf{Bytes}.\\
	\hline
	\texttt{3} + \textbf{Start} & \textbf{Crc}: suma kontrolna obliczana od pola \textbf{Start} do pola \textbf{Ładunek} włącznie. \\
	\hline
\end{longtable}