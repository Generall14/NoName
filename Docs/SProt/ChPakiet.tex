\newpage

\section{Pakiet}
\label{Sec_pakiet}

\begin{longtable}{|p{\BWIDTH}|p{\OWIDTH}|}
	\hline
	\textbf{Bajt} & \textbf{Opis}\\ 
	\hline
	\hline
	\texttt{0} & \textbf{Start}: stała wartość rozpoczynająca pakiet, zawsze równa \texttt{0x5A}\\
	\hline
	\texttt{1} & \\
			& \begin{tabular}{|p{\IBWIDTH}|p{\IOWIDTH}|}
				\hline
				\textbf{Bit} & \textbf{Opis}\\ 
				\hline
				\hline
				0 -- 6 & \textbf{Bytes}: liczba bajtów ładunku, od 0 do 64.\\
				\hline
				7 & \textbf{CmdH}: najstarszy bit komendy.\\
				\hline
			\end{tabular}\\
	\hline
	\texttt{2} & \textbf{CmdL}: młodszy bajt komendy.\\
				& \textbf{Cmd = CmdL | CmdH<<8}: Komanda pakietu.
	 \\
	\hline
	\texttt{3} ... & \textbf{Ładunek}: ładunek komendy, liczba bajtów w ładunku zapisana w polu \textbf{Bytes}.\\
	\hline
	\texttt{3} + \textbf{Start} & \textbf{Crc}: suma kontrolna obliczana od pola \textbf{Start} do pola \textbf{Ładunek} włącznie. \\
	\hline
\end{longtable}

Maksymalny czas nadawania pakietu to $1 ms$ (od nadania bajtu Start), po tym czasie dane zostaną odrzucone.

Nieznany numer komendy, nieprawidłowy rozmiar ładunku, błędna wartość pola Start oraz nieprawidłowa wartość sumy kontrolnej spowodują odrzucenie komendy.

Transmisja przebiega w trybie master-slave.

\subsection{Suma kontrolna}
\label{SubSec_sumaKontrolna}

Suma kontrolna obliczana jest od pola \textbf{Start} do pola \textbf{Ładunek} włącznie.

CRC-8:
\begin{itemize}
	\item $init = 0x55$
	\item $poly = 0xCF$
\end{itemize}
