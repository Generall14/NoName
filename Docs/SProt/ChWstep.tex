\section{Wstęp}
\label{Sec_Wstep}

\subsection{Zawartość dokumentu}
\label{SubSec_zawartosc}
Niniejszy dokument zawiera specyfikację protokołu \NAME wykorzystywanego gdzieś tam. Specyfikuje warstwę łącza danych, warstwa fizyczna zależy od wykorzystanego medium, warstwy wyższe zależne są od chwilowego widzi mi się.

\subsection{Notacje}
\label{SUbSec_notacje}
W dokumencie stosuje się następujące notacje liczb:
\begin{itemize}
	\item 17 -- liczba dziesiętna,
	\item 0x11 -- liczba szesnastkowa,
	\item 0b0001\_0001 -- liczba binarna, najstarszy bit umieszczony po lewej stronie, \_ separator umieszczany w dowolnym miejscu,
	\item wartości są domyślnie wyrównywane do najmniej znaczących bitów (np. jeżeli z opisu danych wynika, że w polu o rozmiarze 16 bitów znajduje się wartość 10 bitowa -- oznacza to, że wartość znacząca jest umieszczona na najmłodszych 10 bitach sekcji 16 bitowej).
\end{itemize}
