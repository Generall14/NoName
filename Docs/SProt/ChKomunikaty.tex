\newpage

\section{Komunikaty}
\label{Sec_komunikaty}

\subsection{Hello \texttt{(0x100)}}
\label{hello}
Komunikat testowy, urządzenie docelowe powinno odpowiedzieć komendą \nameref{reHello}. Nie zawiera ładunku.

\subsection{ReHello \texttt{(0x000)}}
\label{reHello}
Komenda odsyłana po odebraniu \nameref{hello}.
\begin{longtable}{|p{\BWIDTH}|p{\OWIDTH}|}
	\hline
	\textbf{Bajt} & \textbf{Opis}\\ 
	\hline
	\hline
	\texttt{0} ... & \textbf{Desc}: ciąg tekstowy ASCII z nazwą urządzenia.\\
	\hline
\end{longtable}

\subsection{GetSec \texttt{(0x101)}}
\label{getSec}
Zapytanie o odczyt sekcji danych, urządzenie powinno odpowiedzieć \nameref{reGetSec}.
\begin{longtable}{|p{\BWIDTH}|p{\OWIDTH}|}
	\hline
	\textbf{Bajt} & \textbf{Opis}\\ 
	\hline
	\hline
	\texttt{0} & \textbf{Sec}: numer sekcji danych do odczytu.\\
	\hline
	\texttt{1 -- 2} & \textbf{Offset}: offset odczytu z sekcji danych.\\
	\hline
\end{longtable}

\subsection{ReGetSec \texttt{(0x001)}}
\label{reGetSec}
Odpowiedź na \nameref{getSec}, zawiera odczytane dane sekcji.
\begin{longtable}{|p{\BWIDTH}|p{\OWIDTH}|}
	\hline
	\textbf{Bajt} & \textbf{Opis}\\ 
	\hline
	\hline
	\texttt{0} & \textbf{Sec}: numer odczytywanej sekcji danych.\\
	\hline
	\texttt{1 -- 2} & \textbf{Offset}: offset początku odczytu.\\
	\hline
	\texttt{3} ... & \textbf{Data}: odczytane dane sekcji. Przesłana zostanie liczba bajtów aż do zakończenia sekcji danych lub do zapełnienia komendy (61 bajtów). Pole \textbf{Data} będzie puste gdy podany zostanie nieistniejący numer sekcji lub \textbf{Offset} będzie wskazywał poza zakres danej sekcji.\\
	\hline
\end{longtable}

\subsection{SetSec \texttt{(0x102)}}
\label{setSec}
Żądanie zapisu do sekcji danych, urządzenie powinno odpowiedzieć komendą \nameref{reSetSec}.
\begin{longtable}{|p{\BWIDTH}|p{\OWIDTH}|}
	\hline
	\textbf{Bajt} & \textbf{Opis}\\ 
	\hline
	\hline
	\texttt{0} & \textbf{Sec}: numer nadpisywanej sekcji danych.\\
	\hline
	\texttt{1 -- 2} & \textbf{Offset}: offset początku zapisu.\\
	\hline
	\texttt{3} ... & \textbf{Data}: dane nadpisywanej sekcji, maksymalnie 61 bajtów.\\
	\hline
\end{longtable}

\subsection{ReSetSec \texttt{(0x002)}}
\label{reSetSec}
Odpowiedź na \nameref{setSec}, zawiera status zapisu sekcji danych.
\begin{longtable}{|p{\BWIDTH}|p{\OWIDTH}|}
	\hline
	\textbf{Bajt} & \textbf{Opis}\\ 
	\hline
	\hline
	\texttt{0} & \textbf{Sec}: numer nadpisywanej sekcji danych.\\
	\hline
	\texttt{1 -- 2} & \textbf{Offset}: offset początku zapisu.\\
	\hline
	\texttt{3} & \textbf{Status}: status zapisu danych:\\
	& \begin{tabular}{|p{\IBWIDTH}|p{\IOWIDTH}|}
		\hline
		\textbf{Wartość} & \textbf{Opis}\\ 
		\hline
		\hline
		0x00 & \textbf{Ok}: zapis przebiegł pomyślnie.\\
		\hline
		0x01 & \textbf{ReadOnly}: sekcja tylko do odczytu, dane nie zostały zmienione.\\
		\hline
		0x02 & \textbf{OutOfRange}: wskazane dane (\textbf{Offset}, długość danych) wybiegały poza zakres sekcji, dane nie zostały zmienione.\\
		\hline
		0x03 & \textbf{UnknownSec}: brak sekcji danych o wskazanym numerze, dane nie zostały zmienione.\\
		\hline
	\end{tabular}\\
	\hline
\end{longtable}